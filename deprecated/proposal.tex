\subsection{Advanced Deep-Learning for COVID-19}

We discuss two objectives that are closely associated with Aim 3. This includes the introduction of Deep learning-based forecasting of COVID-19 as well as the introduction of Generative Adversarial Network (GANs) for generating new model-based data for COVID-19. Combined with our other Aims, we will develop a novel system for analyzing disease spread. Notably, GANs form a cutting edge, new development in deep learning that were introduced in different applications successfully since 2014 [15].

Objective 1. Deep Learning-based forecasting of COVID-19. There is increased recognition of the importance of deep learning in data-driven discovery across a broad range of applications \TODO{[16]–[18]}. Deep learning uses multiple layers of neural networks to consecutively extract multi-level features from the raw input. Due to the advent of fast GPUs, deep learning has become a formidable option in many optimization problems. The advantage of deep learning is the ability to search for features that are inherent in the underlying dataset without explicitly identifying them. This makes them adaptable for time series analysis as applied to COVID-19 and other infectious diseases. They offer us the opportunity to include not only the time-dependent variables but also additional health and socio-economic risk factors that influence the prediction. 

In this Aim, we propose to leverage deep learning and integrate risk factors that will ultimately lead to better predictions that obviously could have an impact on scheduling scarce resources. Furthermore, the inclusion of risk factors can be used to identify population sectors that are most vulnerable. The inclusion of micro and macro data is essential to allow for prediction on various time scales and geographically and socio-economically diverse populations. The system will be designed in such a fashion that not only quasi-real time predictions can be made, but that encourages the inclusion of “what-if-scenarios” helping policymakers to decide what actions need to be taken in which scenario. Hence, the overall framework must be easily reusable by the community and is discussed in more detail in Aim 3. 

Preliminary investigation. We are well-positioned to develop such a system as our preliminary investigation (manuscript under preparation [19]) has shown that we can create new predictions rapidly through deep learning while integrating many different risk factors too, for example, deal with new trends, under-reporting [20], or data corrections as we have seen in COVID-19 data reporting [21]. In our pilot study, we have shown that deep learning can be applied to complex COVID-19 pandemic data analysis that can both capture the dynamics to create a deep learning operand and allow predictions. Using county-level CDC BRFSS risk factors, Census-based demographic and socioeconomic variables, and February-March 2020 time-series data on COVID-19 data on cases and deaths from 110 US cities, we used an extensive parameter sweep to identify deep learning operators with very accurate predictions for not only the overall prediction of the cases and deaths but also in regards to their geographical distributions. Fig 1. shows the accuracy of our modeling and forecasting results with error estimates summed up over all locations and contrasting them to the real and predicted data. 
Our pilot incorporates (a) health and behavioral risk factors, (b) socioeconomic parameters, (c) an adaptable and sophisticated time series analysis incorporating multivariate and multistep [22] input and outputs (e.g. in our preliminary analysis steps refer to days), and  (d) and allows inclusion of updated, different, and new data sources [23][24][25][26][27][28]. From this and other data sources our Aim 1 will create temporal and spatial useful datasets for our deep learning framework. The suitability of our deep learning framework for utilizing information for covariates was demonstrated in this preliminary work as many risk factors increase the accuracy of our predictions, as shown in Fig 5. Our analysis “learned” from our experiments that the average accuracy of the model increases while incorporating parameters such as county-level availability of hospitals, the percentage of African Americans or seniors, the percentage of insured people, among others. 

 
Fig. 1: Agglomerated deep learning fits of 110 US cities for COVID-19 case and death data in the US from Feb. 1 to May 25, 2020, with predictions 2 weeks out and showing weekly structure (to showcase the sum over the error of all 110 Cities the  axis is normalized and not shown in \# count)
	
 
Fig. 2: Distribution of the accuracy of the prediction upon inclusion of health factors by county.


In this project, we will further use our integrative capability to include more datasets describing risk factors in particular but social and environmental determinants of health in general, and determine their combinations that are relevant for our model predictions. This will allow us to model the disease dynamics in a much broader and powerful framework as we not only focus on time-dependent and spatial instances of the disease, but also on the underlying population characteristics. To establish the feasibility of our approach, we have in our second pilot analysis compared them to epidemiological models with empirical fits [29]. The time series were 100 days long and a multi-layered Long Short-Term Memory (LSTM) [30] recurrent network was used. It differed by learning not only from the demographics (fixed data for each city) as well as time-dependent data but also by integrating the population model for the underlying prediction as shown in Fig. 2. Such model predictive integration capability is important in any application with multiple time scales. For example, it will allow us to integrate multiscale time effects into the forecast, which could address the combination of general forecasts and the next time steps of choice such as days, weeks, months, or longer. For this pilot, we used 37 of the 110 cities with reliable empirical (not deep learning) fits for the case and death data up to April 15, 2020 [29]. Our pilot identified a sophisticated single deep learning time evolution operator that can describe these 37  separate datasets and smooth fitted data leads to very accurate deep learning descriptions as depicted in both Fig. 1 and 3. In our planned work, we will expand this surrogate for an empirical fit to a surrogate for a sophisticated epidemiological simulation combining all our aims and giving the community a new novel tool for analysis (Fig. 4). The new deep learning operators thus developed will become an essential part of the model prediction toolkit of an epidemiologist or disease modeler and pave the way for future epidemiologic studies. 
Our next objective will allow the researchers to generate realistic samples that mimic real statistical properties of a given population, in this case, corresponding to COVID-19 cases and deaths. As we will likely identify further and newer parameters and networks based on those parameters, we will also integrate them into a GAN in order to improve their contributions, and go on to identify accurate model predictions when input data may not be available but at the same time, the output sample will continue to be realistic. 

 
Fig. 3: Deep learning fits to empirical COVID-19 data descriptions with 37 separate results shown as summed over cities. The cases and death were learned together on each separate fit (to showcase the sum over the error of all 37 Cities the  axis is normalized and not shown in \verb|#| count) 	 
Fig. 4: AICov Architecture

\section{Dessimination}

Implementation and distribution of data products and open-source software packages 
We will make the software and data APIs and sources freely available to the community. Tutorials for guiding the users of the platform will be made available so that programs can be easily recreated or reproduced. The code will be managed in a GitHub.com and replications will be hosted in the organizations as backups. 
Impact on other disciplines: Our main focus is to make this framework available to others within the NIH community. In 2004, Modeling of Infectious Disease Agent Study (MIDAS) [40] was established by the National Institute of General Medical Sciences (NIGMS) as a collaborative network of research scientists who use computational, statistical and mathematical models to understand infectious disease dynamics and assist the nation to prepare for, detect and respond to infectious disease threats. As a member of the MIDAS Network, Dr. Pyne will share the results and platforms developed by this project with his fellow-members.
We intend to donate the application of COVID-19 data to MLPerf (with representation from major vendors) [41], [42] as a contribution to evaluate deep learning systems. It could serve as an essential contribution to two working groups: Time Series led by Huang and Science Data which was just set up and is led by Fox (PI of this project) and Hey (chief data scientist of UK science instruments) [43]. Hence this work could be reused as a modern and important benchmark for machine learning services and hardware. Furthermore, lessons learned from this effort can also be applied to other applications [44], [45], [46], [43], [47]. 
Preliminary Results: We have rich experience in managing large scale open source projects on GitHub with many contributors as we see in Cloudmesh and other projects we execute at IU. For example, Cloudmesh has been utilized by more than 200 students and researchers and received direct contributions from 70 that have been integrated [38]. IU has rich experience in big data management and has worked on many big data projects. Through the reuse of these activities, we can reduce the overall cost of the proposal, while at the same time adding functionality that would otherwise not be possible. Dr. Fox is leading the effort of MLPerf Time Series benchmark, while Dr von Laszewski is leading the NIST NBDIF Interfaces working group allowing easy dissemination and adoption of Health and other discipline members. Dr. Pyne is a member of the MIDAS Network of NIGMS, and will share the results and modeling frameworks of the project with his fellow-members.
Timeline (Table 1)

