\documentclass{article}
\usepackage{fullpage}
\usepackage[utf8]{inputenc}

\title{Review Response JDS2007-001RA0}

\date{Revision due: 2020-11-24}

\begin{document}

\maketitle

\parindent 0pt We are happy to resubmit our paper and are very thankful for the great reviews we received. Thanks to the review comments, the quality of the paper has significantly improved. 

In particular, we addressed the following:

\begin{enumerate}

\item The paper has been checked multiple times by multiple native English speakers.

\item The organization of the paper has been improved while focusing entirely on the deep learning framework. Several sections and images that distracted the reader from the paper's primary focus (which was evident by the careful reviews) were removed.

\item Reviewer 1: The method has been described using a recipe approach without giving the rationality behind it. There are too many predictive models for COVID-19. What are the advantages of the proposed method? The authors should explain their competitiveness, at least from the following two perspectives. We addressed these issues through the next two items.

\item Reviewer 1: First, compared with the very classical SIR/SEIR and their improved models, what are the advantages of the proposed model? 

    A new section has been added to discuss the advantages and disadvantages.

\item Reviewer 1: Is the accuracy improved significantly? 

    While using the real-time data our accuracy is very good as it, for example accounts for the weekly fluctuations that are not covered by other prediction models. Moreover, the point is not only a single model in time, but that the model adapts itself over time while new data is added. This has been pointed out in the paper.

\item Reviewer 1: Second, it seems that the data used are all very structured, what about regression model framework? 

    The beauty of this deep learning algorithm is that it learns the trends and performs well even if the data is not smoothed with regression or a seven-day average. Instead, it even learns the weekly fluctuations which we point out in the revised version of the paper.

\item Reviewer 1: Is the performance competitive? 

    The performance is very competitive and an explanation is added to the conclusion. 

\item Reviewer 1: The current deep learning framework is lack of interpretation of the investigated risk factors.

    We added the following section:

While this study focuses on if and which risk factors can improve the predictions, we do not focus on why. Such an analysis could provide insights into causal relationships. For example,  information such as health insurance and the number of hospitals and beds are often related to the quality of services that affect early detection and treatment. Furthermore, population parameters such as density can determine the rate of spread and the number of cases. Similarly, distributions of comorbidities in a given population such as physical health, cholesterol screening, diabetes, etc., are found to be insightful in predicting outcomes. Identifying such risk factors and correlating them to the prediction accuracy provides valuable candidates for future studies.

\item Reviewer 2: The manuscript is in a rather preliminary shape, and needs significant work before being in a publishable shape.  

    We believe we have addressed this issue by addressing all reviewer's comments and more.

\item Reviewer 2: Section 2.2 does not look sufficient as a subsection.  It should be either expanded to include more details, or combined with Section 2.1.3.  

    The sections have been improved/combined, and a section from later on in the paper was added.

\item Page 5, under "Data Source Integration", what does "we include integrate data integration capabilities" mean?

    The sentence has been corrected.

\item Reviewer 2: 4.  Please consider writing the requirements on Pages 5 and 6 into structured paragraphs.

    The requirements have been rewritten as paragraphs. Also, we added a numbered requirement tag to them and kept the high-level name of the requirement for increased readability and structure.

\item Reviewer 2: Item 5., 6., 7., 8., 9., and 10. Have been fixed by excluding the text related to them from the paper as we found they distracted too much from our main contributions.

\item Reviewer 2: 12.  In Section 2.1,  the activation function is the sigmoid function.  In Section 6, RELU was used.

    Thanks for detecting it. We used in both approaches sigmoid only in the last layer and corrected it in the paper

\item Reviewer 2: 13.  Please consider adding labels for the yaxes in Figure 10 and Figure 11.

    All images now have information about the axes.

\item Reviewer 2: 14.  Page 19, please fix the broken link to reference.

    Broken links were fixed.

\item Reviewer 2: 15.  Sections 7 and 8 are rather redundant.  The language does not read clear.  I suggest moving these two sections, and make Section 5 and 6 a more complete story.

    Thanks for this suggestion. To better showcase our arguments we have done the following: (a) most of the theory was moved into the same section in order to reduce redundancy, (b) we introduced a number of Questions that we analyze within this report to provide additional structure and delineate the sections which answer each question, (c) a more comprehensive explanation has been added the later sections. Together they provide a more complete story.

\item Reviewer 2: 16.  Please consider making the code publicly available at GitHub. 

    We intend to make the code public after publication. More than 30K lines of underlying infrastructure code and documentation is already available on GitHub and citations have been added.

\end{enumerate}

\end{document}

